\include{MeijeCours}

\newboolean{versionprof}  % Cette variable booléenne permet le contrôle de la compilation selon la version professeur (avec les preuves, commentaires pédagogiques etc...) si true ou selon la version élève (de ''base'') si false.
\setboolean{versionprof}{false}

% =============================================================================================
%                                                                                      Début du texte 
% =============================================================================================

\author{Spé NSI - Lycée du parc}  % Pour spécifier l'auteur qui sera affiché sous le titre
\title{Correction des algorithmes}  % Titre 
\date{Année 2020-2021} %en utilisant \today on obtiendra la date courante lors de la production du document
\renewcommand{\thesection}{\Roman{section}}  % permet de numéroter les sections en chiffres romains
\pagestyle{fancy}

\lhead{NSI}   %  Indiquer la classe ici  
\rhead{Correction des algorithmes}		% Indiquer la date pour rendre le devoir ici


\fancyfoot[C]{\thepage} % bas de page : centre par exemple:  - \thepage\ -
\fancyfoot[L]{} % bas de page : gauche
\fancyfoot[R]{} % bas de page : droite
\newcommand{\p}[1]{ \left( #1 \right)}   %parenthèse
\newcommand\abs[1]{|{#1}|}

\begin{document}

\maketitle  %produit le titre conformément aux instructions données par \author, \date ...etc
\thispagestyle{empty}
% ------------------------------------------------   Introduction   -------------------------------------------
\vskip 1cm
\renewcommand{\abstractname}{Introduction\hfill}
\begin{abstract} 

On poursuit ici l'étude théorique des algorithmes entreprise dans le chapitre traitant de la complexité.

On se pose maintenant la question de savoir si un algorithme donné répond bien au problème qu'il est censé traité dans sa \emph{spécification}.
Il se pose alors deux grandes questions : 
\begin{enumerate}[-]
  \item Se termine-t-il ? C'est la question de la \emph{terminaison}.
  \item Résout-il bien le problème qu'il est censé traiter ? C'est la question de la \emph{correction}\footnote{On parle parfois de correction partielle plutôt que de correction. La correction totale étant la conjonction de la terminaison et de la correction partielle. }.
 
\end{enumerate}

Le but de ce chapitre est d'introduire les méthodologies qui permettent de traiter ces problèmes.
\end{abstract} 


\section{Terminaison}

Pour un algorithme ou une partie d'algorithme qui ne comporte pas de boucles ou seulement des boucles inconditionnelles, la question de la terminaison ne se pose, a priori, pas. 
Le cas des boucles conditionnelles est plus délicat : la condition est censée être vraie au départ (sinon c'est du code mort) et cette même condition doit finir par être fausse sinon les itérations ont lieu indéfiniment.

\exercice{}

Laquelle de ces deux boucles ne se termine pas ?

\begin{minipage}{0.5\linewidth} %début texte
\begin{minipage}{0.8\linewidth}
\begin{lstlisting}
x = 0
while x >= 0:
    x += 1

\end{lstlisting}
\end{minipage}\,
\begin{minipage}{0.05\linewidth}
\fbox{\thecode}\addtocounter{code}{1}
\end{minipage}
 
\end{minipage}% fin texte
\hfill 
\begin{minipage}{0.5\linewidth} 
\begin{minipage}{0.8\linewidth}
\begin{lstlisting}
x = 10
while x >= 0:
    x = x - 1

\end{lstlisting}
\end{minipage}\,
\begin{minipage}{0.05\linewidth}
\fbox{\thecode}\addtocounter{code}{1}
\end{minipage}

\end{minipage}

\lignes{8}

\vfill

{\bf Vocabulaire : } On appelle {\bf itération} d'une boucle {\bf une} exécution des instructions qui figure dans le corps de la boucle.
\medskip

{\bf Démonstration de la terminaison}
\medskip

Pour démontrer qu'une boucle conditionnelle (\lstinline{while}) se termine, il suffit de déterminer une quantité exprimée à l'aide des variables de l'algorithme qui  {\bf reste positive et entière} tout au long de l'exécution de la boucle et {\bf décroit strictement} à chaque itération.
Comme il n'existe pas de suite infinie à valeurs dans $\N$ qui soit strictement décroissante cela prouve alors que le nombre d'itérations est fini.

On appelle souvent {\bf variant} de la boucle une telle quantité\footnote{Certain auteurs utilisent aussi \emph{convergent} ou \emph{expression de terminaison}.}.

 
 \exercice{} Montrer que l'algorithme de division euclidienne dans $\N$ se termine (avec  $b>0$) :
 \begin{minipage}{0.5\linewidth}
 \begin{lstlisting}
def division_euclidienne(a, b):
    q = 0
    r = a
    while r >= b :
        r = r - b
        q = q + 1
    return (q, r)
 
 \end{lstlisting}
 \end{minipage}
 \hskip 1cm\hfill 
 \begin{minipage}{0.5\linewidth} 
 \lignespar{7}{80}
 \end{minipage}
 \lignes{7}
 \exercice{} Montrer que l'algorithme de recherche séquentielle dans une liste se termine :
 \begin{minipage}{0.5\linewidth}
 \begin{lstlisting}
def appartient(x, L):
    n = len(L)
    i = 0

    while i < n and L[i] != x:
        i += 1

    return i != n 
 \end{lstlisting}
 \end{minipage}
 \hskip 1cm
 \hfill 
 \begin{minipage}{0.5\linewidth} 
 \lignespar{7}{80}
 \end{minipage}
\vskip -0.3cm
 \lignes{9}

\exercice{}

Les deux fonctions ci-dessous calculent le produit de deux entiers positifs. Justifier qu'elles terminent à l'aide d'un variant.

\begin{minipage}{0.4\linewidth} 
 \begin{lstlisting}
def prod1(a, b):
    S = 0
    while b > 0:
        S = S + a
        b = b - 1
    return S
\end{lstlisting}

\end{minipage}
\hfill 
\begin{minipage}{0.47\linewidth} 
\begin{lstlisting}
def prod2(a, b):
    S = 0
    while b > 0:
        if b%2 == 1:
            S = S + a
        b = b//2  # quotient / 2 
        a = a + a
    return S
    

\end{lstlisting}

\end{minipage}

\lignes{10}

\entrainement

Les deux fonctions suivantes calculent le nombre de chiffres de l'écriture décimal de \verb+n+ pour $n>0$. Justifier qu'elles terminent à l'aide d'un variant.

\begin{minipage}{0.45\linewidth} 
 \begin{lstlisting}
def nbChiffres1(n):
    k = 0
    while n > 0:
        n = n // 10
        k = k + 1
    return k
\end{lstlisting}

\end{minipage}
\hfill 
\begin{minipage}{0.45\linewidth} 
\begin{lstlisting}
def nbChiffres2(n):
    p = 1
    k = 0
    while p < n + 1:
        p = p*10
        k = k + 1
    return k
\end{lstlisting}

\end{minipage}


\section{Correction}

Pour montrer la correction d'un algorithme, les difficultés se posent dans les boucles (quel qu'en soit le type, conditionnelles ou inconditionnelles). On utilise un {\bf invariant de boucle} c'est une propriété qui reste vraie tout au long des itérations et qui, à la fin, donne le résultat attendu.

\exercice{} Montrer que la fonction \verb+puissance+ calcule bien $x^{n}$.

\begin{minipage}{0.45\linewidth} %début texte
\begin{minipage}{0.9\linewidth}
\begin{lstlisting}
def puissance(x, n):
    p = 1
    for k in range(n):
        p = p*x
    return p

\end{lstlisting}
\end{minipage}
\hskip 1cm
 
\end{minipage}% fin texte
\hfill 
\begin{minipage}{0.5\linewidth} 
\lignespar{5}{90}
\end{minipage}

\lignes{6}
 

\exercice{} Montrer que l'algorithme de division euclidienne dans $\N$ est correct (avec  $b>0$).

\begin{minipage}{0.5\linewidth} %début texte
\begin{minipage}{0.9\linewidth}
\begin{lstlisting}
def division_euclidienne(a, b):
    q = 0
    r = a
    while r >= b :
        r = r - b
        q = q + 1
    return (q, r)
\end{lstlisting}
\end{minipage}\,
 
\end{minipage}% fin texte
\hfill 
\begin{minipage}{0.5\linewidth} 
\lignespar{7}{85}
\end{minipage}

\lignes{11}

\exercice{}

Montrer que l'algorithme de recherche du maximum dans une liste est correct :

\begin{minipage}{0.5\linewidth} %début texte
\begin{minipage}{0.9\linewidth}
\begin{lstlisting}
def maximum(L):
    n = len(L)
    maxi = L[0]
    for x in L[1:]:
        if x > maxi: 
            maxi = x
    return maxi
\end{lstlisting}
\end{minipage}\,
 
\end{minipage}% fin texte
\hfill 
\begin{minipage}{0.5\linewidth} 
\lignespar{7}{85}
\end{minipage}

\lignes{10}
\eject
\exercice{}

Montrer que l'algorithme de recherche séquentielle dans une liste est correct :

\begin{minipage}{0.5\linewidth} %début texte
\begin{minipage}{0.9\linewidth}
\begin{lstlisting}
def appartient(L, x):
    n = len(L)
    i = 0
    while i < n and x != L[i]:
        i += 1
    return i < n
\end{lstlisting}
\end{minipage}\,
 
\end{minipage}% fin texte
\hfill 
\begin{minipage}{0.5\linewidth} 
\lignespar{6}{85}
\end{minipage}

\lignes{12}
\exercice{}
Les deux fonctions ci-dessous calculent le produit de deux entiers positifs. Justifier leur correction à l'aide d'un invariant.

\begin{minipage}{0.4\linewidth} 
 \begin{lstlisting}
def prod1(a, b):
    S = 0
    while b > 0:
        S = S + a
        b = b - 1
    return S
\end{lstlisting}

\end{minipage}
\hfill 
\begin{minipage}{0.47\linewidth} 
\begin{lstlisting}
def prod2(a, b):
    S = 0
    while b > 0:
        if b%2 == 1:
            S = S + a
        b = b//2  # quotient / 2 
        a = a + a
    return S
    

\end{lstlisting}

\end{minipage}

\lignes{15}



\entrainement{}

\'Ecrire une fonction qui calcule la somme des éléments d'un tableau et montrer sa correction avec un invariant.

\entrainement{}

\begin{minipage}{0.7\linewidth} %début texte
\medskip

 On consid\`{e}re le tableau ci-contre o\`{u} seul l'\'{e}l\'{e}ment \`{a}
l'intersection de la premi\`{e}re ligne et de la troisi\`{e}me
colonne est un signe \og $-$\fg{}, les autres \'{e}l\'{e}ments
\'{e}tant des signes \og $+$\fg{}.
On effectue des transformations en changeant tous les signes d'une
ligne, d'une colonne ou d'une diagonale quelconque. 

Est-il possible
qu'apr\`{e}s un certain nombre de telles op\'{e}rations on se retrouve avec
un tableau o\`{u} tous les \'{e}l\'{e}ments sont des \og +\fg{} ?

\end{minipage}% fin texte
\hfill 
\begin{minipage}{0.3\linewidth} 
\begin{center}
\begin{tabular}[t]{|c|c|c|c|}
  \hline
  % after \\: \hline or \cline{col1-col2} \cline{col3-col4} ...
  + & + & $-$ & + \\
  \hline
  + & + & + & + \\
  \hline
  + & + & + & + \\
  \hline
  + & + & + & + \\
  \hline
\end{tabular}\\
\end{center}
\end{minipage}
\bigskip



\section{Retour sur la recherche dichotomique}


On rappelle l'algorithme de recherche dichotomique dans une liste triée : 


\bigskip

\begin{minipage}{0.9\linewidth}
\begin{lstlisting}[escapeinside =;;]
def rech_dicho(L, x):
    """ La liste L est suppos;\it é;e tri;\it é;e. 
    Renvoie un indice d'un ;\it é;l;\it é;ment de L de valeur x 
    si x est pr;\it é;sent dans L et None sinon """
    
    g, d = 0, len(L) -1 

    while g <= d:
        m = (g + d) // 2
        if L[m] < x:
            g = m + 1
        elif L[m] > x:
            d = m - 1 
        else:
            return m
            
    return None

\end{lstlisting}

\end{minipage}\,
\begin{minipage}{0.05\linewidth}
\fbox{\thecode}\addtocounter{code}{1}
\end{minipage}

\exercice{}

Faire fonctionner l'algorithme à la main pour :
\setlength{\columnseprule}{0pt} 
\begin{multicols}{2}  

\begin{enumerate}[1.]
  \item \verb+L=[1, 3, 5, 7, 10]+ et \verb+x+ =7. 
  \item \verb+L=[1, 3, 5, 7, 10, 17]+ et \verb+x+ =2. 
\end{enumerate}
\end{multicols} 


\lignes{14}
\exercice{}
Montrer la terminaison de l'algorithme de recherche dichotomique.

\lignes{22}

\exercice{}
Montrer la correction de l'algorithme de recherche dichotomique.

\lignes{20}


\section{Les tris}

\exercice{}

On rappelle le code du tri par selection\footnote{On dispose d'une fonction {\bf permute(t, i, j)} qui permute les éléments d'indices i et j du tableau t. }.

Justifier sa terminaison et démontrer sa correction.
\begin{lstlisting}[escapeinside =;;]
def tri_selection2(t):
    for i in range(len(t)):
        j_record = i
        for j in range(i + 1, len(t)):
            if t[j] < t[j_record]:
                j_record = j
        permute(t, i, j_record)

\end{lstlisting}

\lignes{25}

\exercice

On rappelle le code du tri par insertion. 

Justifier sa terminaison et démontrer sa correction.

\begin{lstlisting}[escapeinside =;;]
def insere(t, i):
    """ t est un tableau et i est un indice de t. On suppose
    que t est tri;\it é; jusqu';\it à; la position i-1. La fonction positionne
    l';\it é;l;\it é;ment d'indice i ;\it à; sa place (effet de bord) """

    for j in range(i, 0, -1):
        if t[j - 1] > t[j]:
            permute(t, j, j-1)
        else:
            return

def tri_insertion_en_place(t):
    for i in range(1, len(t)):
        insere(t, i)

\end{lstlisting}


\lignes{30}
\end{document}




